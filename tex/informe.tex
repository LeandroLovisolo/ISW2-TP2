\documentclass[a4paper, 10pt, twoside]{article}

\usepackage[top=1in, bottom=1in, left=1in, right=1in]{geometry}
\usepackage[utf8]{inputenc}
\usepackage[spanish, es-ucroman, es-noquoting]{babel}
\usepackage{setspace}
\usepackage{fancyhdr}
\usepackage{lastpage}
\usepackage{amsmath}
\usepackage{amsfonts}
\usepackage{amsthm}
\usepackage{verbatim}
\usepackage{fancyvrb}
\usepackage{graphicx}
\usepackage{float}
\usepackage{enumitem} % Provee macro \setlist
\usepackage{tabularx}
\usepackage{multirow}
\usepackage{hyperref}
\usepackage{xspace}
\usepackage{qtree}
\usepackage[toc, page]{appendix}


%%%%%%%%%% Constantes - Inicio %%%%%%%%%%
\newcommand{\titulo}{Trabajo Práctico 2}
\newcommand{\materia}{Ingeniería de Software II}
\newcommand{\integrantes}{Izcovich · Lovisolo · Petaccio · Vita}
\newcommand{\cuatrimestre}{Primer Cuatrimestre de 2015}
%%%%%%%%%% Constantes - Fin %%%%%%%%%%


%%%%%%%%%% Configuración de Fancyhdr - Inicio %%%%%%%%%%
\pagestyle{fancy}
\thispagestyle{fancy}
\lhead{\titulo\ · \materia}
\rhead{\integrantes}
\renewcommand{\footrulewidth}{0.4pt}
\cfoot{\thepage /\pageref{LastPage}}

\fancypagestyle{caratula} {
   \fancyhf{}
   \cfoot{\thepage /\pageref{LastPage}}
   \renewcommand{\headrulewidth}{0pt}
   \renewcommand{\footrulewidth}{0pt}
}
%%%%%%%%%% Configuración de Fancyhdr - Fin %%%%%%%%%%


%%%%%%%%%% Miscelánea - Inicio %%%%%%%%%%
% Evita que el documento se estire verticalmente para ocupar el espacio vacío
% en cada página.
\raggedbottom

% Separación entre párrafos.
\setlength{\parskip}{0.5em}

% Separación entre elementos de listas.
\setlist{itemsep=0.5em}

% Asigna la traducción de la palabra 'Appendices'.
\renewcommand{\appendixtocname}{Apéndices}
\renewcommand{\appendixpagename}{Apéndices}

\newcommand{\grafico}[1]{
  \begin{center}
    \includegraphics[width=15cm]{#1}
  \end{center}
}


%%%%%%%%%% Miscelánea - Fin %%%%%%%%%%

\begin{document}


%%%%%%%%%%%%%%%%%%%%%%%%%%%%%%%%%%%%%%%%%%%%%%%%%%%%%%%%%%%%%%%%%%%%%%%%%%%%%%%
%% Carátula                                                                  %%
%%%%%%%%%%%%%%%%%%%%%%%%%%%%%%%%%%%%%%%%%%%%%%%%%%%%%%%%%%%%%%%%%%%%%%%%%%%%%%%


\thispagestyle{caratula}

\begin{center}

\includegraphics[height=2cm]{DC.png} 
\hfill
\includegraphics[height=2cm]{UBA.jpg} 

\vspace{2cm}

Departamento de Computación,\\
Facultad de Ciencias Exactas y Naturales,\\
Universidad de Buenos Aires

\vspace{4cm}

\begin{Huge}
\titulo
\end{Huge}

\vspace{0.5cm}

\begin{Large}
\materia
\end{Large}

\vspace{1cm}

\cuatrimestre

\vspace{4cm}

\begin{tabular}{|c|c|c|}
\hline
Apellido y Nombre & LU & E-mail\\
\hline
Izcovich, Sabrina      & 550/11 & sizcovich@gmail.com\\
Lovisolo, Leandro      & 645/11 & leandro@leandro.me\\
Petaccio, Lautaro José & 443/11 & lausuper@gmail.com\\
Vita, Sebastián        & 149/11 & sebastian\_vita@yahoo.com.ar\\
\hline
\end{tabular}

\end{center}

\newpage

\tableofcontents

\newpage


%%%%%%%%%%%%%%%%%%%%%%%%%%%%%%%%%%%%%%%%%%%%%%%%%%%%%%%%%%%%%%%%%%%%%%%%%%%%%%%
%% Introducción                                                              %%
%%%%%%%%%%%%%%%%%%%%%%%%%%%%%%%%%%%%%%%%%%%%%%%%%%%%%%%%%%%%%%%%%%%%%%%%%%%%%%%

\section{Introducción}

En el presente trabajo se aborda la extensión del diseño presentado en el TP1, debiendo ahora plantear un sistema de mensajes de muchas campañas en a nivel nacional y para muchos clientes. Dada esta nueva situación, aparecen varios stakeholders que, por su colaboración en el proyecto proveyendo logística o servicios, o por su intención de consumo, imponen distintos requerimientos que sumados a los técnicos desembocan en la contemplación de diversos riesgos y atributos de calidad.
Aquí se presenta una planificación de las fases de elaboración y construcción de la metodología UP, asumiendo que la fase inception ya se ha desarrollado. En esta primera etapa se han presentado las ideas para la formulación del proyecto Big Tiza, se han establecido los requerimientos y recursos disponibles, y se ha llevado a cabo el QAW del cual se desprendieron algunos atributos de calidad esperados en el producto final. Como parte de la primera iteración, se identificaron los casos de uso más relevantes y en base a la recopilación de toda esta información se realizó un análisis de riesgos que se utilizó en parte para organizaron los CU en las distintas iteraciones incrementales. Además se diseñó la arquitectura del sistema, realizando una comparación con la del TP1, lo cual motivó algunas comparaciones sobre el contraste entre programming in the small vs. programming in the large.

\newpage

%%%%%%%%%%%%%%%%%%%%%%%%%%%%%%%%%%%%%%%%%%%%%%%%%%%%%%%%%%%%%%%%%%%%%%%%%%%%%%%
%% Plan de proyecto                                                          %%
%%%%%%%%%%%%%%%%%%%%%%%%%%%%%%%%%%%%%%%%%%%%%%%%%%%%%%%%%%%%%%%%%%%%%%%%%%%%%%%

\section{Plan de proyecto}
\newpage

%%%%%%%%%%%%%%%%%%%%%%%%%%%%%%%%%%%%%%%%%%%%%%%%%%%%%%%%%%%%%%%%%%%%%%%%%%%%%%%
%% Casos de uso                                                              %%
%%%%%%%%%%%%%%%%%%%%%%%%%%%%%%%%%%%%%%%%%%%%%%%%%%%%%%%%%%%%%%%%%%%%%%%%%%%%%%%

\section{Casos de uso}
Para comenzar a trabajar sobre la planificación del proyecto se identificó un conjunto de casos de uso que cubriera la mayoría de las funcionalidades que se desprenden de los requerimientos obtenidos. En un principio se los agrupó, tal y como se muestra a continuación, por su funcionalidad. Posteriormente, y en base al análisis de riesgos y a la información destilada de la fase de incepción, se utilizó este refinamiento para la organización y planificación de las iteraciones de elaboración y construcción.

\newpage

%%%%%%%%%%%%%%%%%%%%%%%%%%%%%%%%%%%%%%%%%%%%%%%%%%%%%%%%%%%%%%%%%%%%%%%%%%%%%%%
%% Análisis de riesgo                                                        %%
%%%%%%%%%%%%%%%%%%%%%%%%%%%%%%%%%%%%%%%%%%%%%%%%%%%%%%%%%%%%%%%%%%%%%%%%%%%%%%%

\section{Análisis de riesgos}
\newpage

\end{document}
