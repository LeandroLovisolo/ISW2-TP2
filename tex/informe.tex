\documentclass[a4paper, 10pt, twoside]{article}

\usepackage[top=1in, bottom=1in, left=1in, right=1in]{geometry}
\usepackage[utf8]{inputenc}
\usepackage[spanish, es-ucroman, es-noquoting]{babel}
\usepackage{setspace}
\usepackage{fancyhdr}
\usepackage{lastpage}
\usepackage{amsmath}
\usepackage{amsfonts}
\usepackage{amsthm}
\usepackage{verbatim}
\usepackage{fancyvrb}
\usepackage{graphicx}
\usepackage{float}
\usepackage{enumitem} % Provee macro \setlist
\usepackage{tabularx}
\usepackage{multirow}
\usepackage{hyperref}
\usepackage{xspace}
\usepackage{qtree}
\usepackage[toc, page]{appendix}


%%%%%%%%%% Constantes - Inicio %%%%%%%%%%
\newcommand{\titulo}{Trabajo Práctico 2}
\newcommand{\materia}{Ingeniería de Software II}
\newcommand{\integrantes}{Izcovich · Lovisolo · Petaccio · Vita}
\newcommand{\cuatrimestre}{Primer Cuatrimestre de 2015}
%%%%%%%%%% Constantes - Fin %%%%%%%%%%


%%%%%%%%%% Configuración de Fancyhdr - Inicio %%%%%%%%%%
\pagestyle{fancy}
\thispagestyle{fancy}
\lhead{\titulo\ · \materia}
\rhead{\integrantes}
\renewcommand{\footrulewidth}{0.4pt}
\cfoot{\thepage /\pageref{LastPage}}

\fancypagestyle{caratula} {
   \fancyhf{}
   \cfoot{\thepage /\pageref{LastPage}}
   \renewcommand{\headrulewidth}{0pt}
   \renewcommand{\footrulewidth}{0pt}
}
%%%%%%%%%% Configuración de Fancyhdr - Fin %%%%%%%%%%


%%%%%%%%%% Miscelánea - Inicio %%%%%%%%%%
% Evita que el documento se estire verticalmente para ocupar el espacio vacío
% en cada página.
\raggedbottom

% Separación entre párrafos.
\setlength{\parskip}{0.5em}

% Separación entre elementos de listas.
\setlist{itemsep=0.5em}

% Asigna la traducción de la palabra 'Appendices'.
\renewcommand{\appendixtocname}{Apéndices}
\renewcommand{\appendixpagename}{Apéndices}

\newcommand{\grafico}[1]{
  \begin{center}
    \includegraphics[width=15cm]{#1}
  \end{center}
}


%%%%%%%%%% Miscelánea - Fin %%%%%%%%%%

\begin{document}


%%%%%%%%%%%%%%%%%%%%%%%%%%%%%%%%%%%%%%%%%%%%%%%%%%%%%%%%%%%%%%%%%%%%%%%%%%%%%%%
%% Carátula                                                                  %%
%%%%%%%%%%%%%%%%%%%%%%%%%%%%%%%%%%%%%%%%%%%%%%%%%%%%%%%%%%%%%%%%%%%%%%%%%%%%%%%


\thispagestyle{caratula}

\begin{center}

\includegraphics[height=2cm]{DC.png} 
\hfill
\includegraphics[height=2cm]{UBA.jpg} 

\vspace{2cm}

Departamento de Computación,\\
Facultad de Ciencias Exactas y Naturales,\\
Universidad de Buenos Aires

\vspace{4cm}

\begin{Huge}
\titulo
\end{Huge}

\vspace{0.5cm}

\begin{Large}
\materia
\end{Large}

\vspace{1cm}

\cuatrimestre

\vspace{4cm}

\begin{tabular}{|c|c|c|}
\hline
Apellido y Nombre & LU & E-mail\\
\hline
Izcovich, Sabrina      & 550/11 & sizcovich@gmail.com\\
Lovisolo, Leandro      & 645/11 & leandro@leandro.me\\
Petaccio, Lautaro José & 443/11 & lausuper@gmail.com\\
Vita, Sebastián        & 149/11 & sebastian\_vita@yahoo.com.ar\\
\hline
\end{tabular}

\end{center}

\newpage

\tableofcontents

\newpage


%%%%%%%%%%%%%%%%%%%%%%%%%%%%%%%%%%%%%%%%%%%%%%%%%%%%%%%%%%%%%%%%%%%%%%%%%%%%%%%
%% Introducción                                                              %%
%%%%%%%%%%%%%%%%%%%%%%%%%%%%%%%%%%%%%%%%%%%%%%%%%%%%%%%%%%%%%%%%%%%%%%%%%%%%%%%

\section{Introducción}

En el presente trabajo se aborda la extensión del diseño presentado en el TP1, debiendo ahora plantear un sistema de mensajes de muchas campañas a nivel nacional tanto para servicios públicos como privados. Dada esta nueva situación, aparecen varios stakeholders que, por su colaboración en el proyecto proveyendo servicios, o por su intención de consumo, imponen distintos requerimientos que sumados a los técnicos desembocan en la contemplación de diversos riesgos y atributos de calidad.
Aquí se presenta una planificación de las fases de elaboración y construcción de la metodología UP, asumiendo que la fase inception ya se ha desarrollado. En esta primera etapa se presentan las ideas para la formulación del proyecto Big Tiza, se establecen los requerimientos y recursos disponibles, y se lleva a cabo el QAW del cual se desprendieron algunos atributos de calidad esperados en el producto final. Como parte de la primera iteración, se identifican los casos de uso más relevantes y en base a la recopilación de toda esta información se realiza un análisis de riesgos que se utiliza en parte para la organización de los CU en las distintas iteraciones incrementales. Además se diseña la arquitectura del sistema.

\newpage

%%%%%%%%%%%%%%%%%%%%%%%%%%%%%%%%%%%%%%%%%%%%%%%%%%%%%%%%%%%%%%%%%%%%%%%%%%%%%%%
%% Casos de uso                                                              %%
%%%%%%%%%%%%%%%%%%%%%%%%%%%%%%%%%%%%%%%%%%%%%%%%%%%%%%%%%%%%%%%%%%%%%%%%%%%%%%%

\section{Casos de uso}
Para comenzar a trabajar sobre la planificación del proyecto se identificó un conjunto de casos de uso que cubriera la mayoría de las funcionalidades que se desprenden de los requerimientos obtenidos. En un principio se los agrupó, tal y como se muestra a continuación, por su funcionalidad. Posteriormente, y en base al análisis de riesgos y a la información destilada de la fase de incepción, se utilizó este refinamiento para la organización y planificación de las iteraciones de elaboración y construcción.

\begin{itemize}

\item \textbf{Comunicación}
\begin{itemize}
\item Enviando resultados de campañas a la NSA.
\item Enviando mensajes de campañas por medios seleccionados.
\item Respondiendo mensaje a una pregunta.
\end{itemize}

\item \textbf{Monitoreo}
\begin{itemize}
\item Consultando estado de una campaña.
\item Visualizando el mapa de evolución de las campañas.
\end{itemize}

\item \textbf{Procesamiento}
\begin{itemize}
\item Registrando cantidad de mensajes enviados por campaña.
\item Generando factura mensual para el sector privado.
\end{itemize}

\item \textbf{Gestión}
\begin{itemize}
\item Ingresando los resultados de una campaña.
\item Creando, editando y eliminando grupos de destinatarios.
\item Creando una campaña desde el sector público/privado.
\item Desuscribiendo a un destinatario de una campaña.
\item Reinscribiendo a un destinatario en una campaña de la que se desuscribió.
\item Agregando evaluación manual del resultado de las preguntas.
\item Aceptando campaña privada.
\item Creando cuestionario de campañas de evaluación de campañas.
\item Agregando nuevo usuario proveedor de contenidos.
\item Creando, editando y eliminando usuarios.
\end{itemize}

\item \textbf{Supervisión}
\begin{itemize}
\item Supervisando campañas privadas (supervisar encuestas)?
\item Pidiendo reportes de resultados de una campaña
\item Reportando información anonimizada de grupos de destinatarios.
\end{itemize}

\item \textbf{Seguridad}
\begin{itemize}
\item Autenticando usuario.
\item Guardando datos cifrados en servidores locales.
\end{itemize}
\end{itemize}

\newpage

%%%%%%%%%%%%%%%%%%%%%%%%%%%%%%%%%%%%%%%%%%%%%%%%%%%%%%%%%%%%%%%%%%%%%%%%%%%%%%%
%% Planificación                                                          %%
%%%%%%%%%%%%%%%%%%%%%%%%%%%%%%%%%%%%%%%%%%%%%%%%%%%%%%%%%%%%%%%%%%%%%%%%%%%%%%%

\section{Planificación}

\subsection{Iteraciones}
La división de los CU en las distintas iteraciones y la organización de las mismas estuvo pautada por las necesidades manifestadas por los stakeholders y un incremento de funcionalidad, partiendo del núcleo más importante y revistiendo la aplicación a lo largo del tiempo de lógica menos prioritaria.

\textbf{Primera iteración} [3 semanas]
\begin{enumerate}
\item Guardando datos cifrados en servidores locales.
\item Creando una campaña desde el sector público/privado.
\item Enviando mensajes de campañas por medios seleccionados.
\item Autenticando usuario.
\end{enumerate}

\textbf{Segunda iteración} [3 semanas]
\begin{enumerate}
\item Creando, editando y eliminando usuarios.
\item Consultando estado de una campaña.
\item Registrando cantidad de mensajes enviados por campaña.
\item Creando, editando y eliminando grupos de destinatarios.
\end{enumerate}

\textbf{Tercera iteración} [3 semanas]
\begin{enumerate}
\item Creando cuestionario de campañas de evaluación de campañas.
\item Respondiendo mensaje a una pregunta.
\item Desuscribiendo a un destinatario de una campaña.
\item Reinscribiendo a un destinatario en una campaña de la que se desuscribió.
\end{enumerate}

\textbf{Cuarta iteración} [2 semanas]
\begin{enumerate}
\item Ingresando los resultados de una campaña.
\item Pidiendo reportes de resultados de una campaña.
\item Agregando evaluación manual del resultado de las preguntas.
\end{enumerate}

\textbf{Quinta iteración} [3 semanas]
\begin{enumerate}
\item Aceptando campaña privada.
\item Reportando información anonimizada de grupos de destinatarios.
\item Generando factura mensual para el sector privado.
\item Enviando resultados de campañas a la NSA.
\end{enumerate}

\textbf{Sexta iteración} [2 semanas]
\begin{enumerate}
\item Supervisando campañas privadas (supervisar encuestas)?
\item Visualizando el mapa de evolución de las campañas.
\item Agregando nuevo usuario proveedor de contenidos.
\end{enumerate}


\subsection{Alcance de casos de usos de la primera iteración}
\textbf{Guardando datos cifrados en servidores locales.}

Los datos de los usuarios deben almacenarse encriptados en los servidores locales de su región correspondiente para poder garantizar la confidencialidad de los mismos.

Llamamos datos a los registros en la base de datos que contienen información sobre los residentes de la región. Estos pueden ser datos básicos como por ejemplo (nombre, apellido, dni, número de teléfono) y datos adiciones como si es fumador, si poseen un auto, si les gusta correr, etc.

\textbf{Creando una campaña desde el sector público/privado.}

La creación de una campaña consiste en el ingreso de los datos necesarios para la creación de la misma, es decir, un conjunto de mensajes con el horario en el que deben ser enviados, un conjunto de grupos de destinatarios concernientes al objetivo de la campaña y el canal a través del que se enviarán (SMS, Facebook, etc).

\textbf{Enviando mensajes de campañas por medios seleccionados.}

Se envían los mensajes de cada campaña a sus destinatarios correspondientes en el momento indicado para cada mensaje. Además, dicho mensaje es enviado a través de los canales previamente seleccionados.

\textbf{Enviando mensajes de campañas por medios seleccionados.}

Cada región tiene un sistema de envío de mensajes que periódicamente recorre las campañas asociadas a esa región y envía los mensajes de las mismas cumpliendo su fecha y hora de envío. Además, cada mensaje es enviado por los canales previamente seleccionados a través de la infraestructura propia de cada región.

\textbf{Autenticando usuario.}

Se solicita un usuario y contraseña cuando una persona intenta iniciar una sesión en el sistema para crear una campaña y consultar el estado de una campaña, entre otras. La autenticación también abarca al sector privado dónde tendran acceso sólo a acciones de su sector.

El sistema verifica que las credenciales ingresadas sean válidas, y en caso contrario deniega el acceso al mismo.

\subsection{Tareas CU Primera iteración}
A continuación se detallan las tareas diagramadas para los casos de uso incluidos en la primera iteración con su respectiva estimación de horas.
\\

\begin{tabular}{lp{13cm}l}
	\hline
	CU1 & Guardando datos cifrados en servidores locales & 50h \\
	\hline \\
	T01 & Investigación de fuentes de información disponibles de cada usuario accesible desde el sistema & 4h \\
	T02 & Definición de campos relevantes para almacenar por usuario & 3h \\
	T03 & Diseño de la base de datos & 4h \\
	T04 & Definición de motor de búsqueda a utilizar & 2h \\
	T05 & Definición de método de encriptación & 2h \\
	T06 & Implementación de CU1 & 10h\\
	T07 & Testing de CU1 & 5h \\
	\hline
\end{tabular}

\begin{tabular}{lp{13cm}l}
	\hline
	CU2 & Creando una campaña desde el sector público/privado & 13h \\
	\hline \\
	T01 & Investigar librerías y herramientas  & 5h \\
	T02 & Diseño de la interfaz del usuario & 2h \\
	T03 & Implementación de CU2 & 4h \\
	T04 & Testing de CU2 & 2h \\
	\hline
\end{tabular}

\begin{tabular}{lp{13cm}l}
	\hline
	CU3 &  Enviando mensajes de campañas por medios seleccionados & 23h \\
	\hline \\
	T01 & Investigación de los distintos medios de envío de mensajes & 8h \\
	T02 & Diseño de un servicio EmisorDeMensajes, que en intervalos de tiempo regulares envíe los mensajes programados & 6h \\
	T03 & Implementación de CU3 & 6h \\
	T04 & Testing de CU3 & 3h \\
	\hline
\end{tabular}

\begin{tabular}{lp{13cm}l}
	\hline
	CU4 & Autenticando usuario & 40h \\
	\hline \\
	T01 & Diseño de la interfaz del usuario para autenticación & 1h \\
	T02 & Diseño de los requisitos mínimos de las contraseñas & 1h \\
	T02 & Implementación de CU4 & 1h \\
	T03 & Testing de CU4 & 1h \\
	\hline
\end{tabular}

\subsection{Detalle Primera iteración}
\begin{itemize}
\item \textbf{Identificación:} E1
\item \textbf{Tipo de iteración:} Elaboración
\item \textbf{Cantidad total de horas:}
\item \textbf{Tareas:}
\begin{enumerate}
\item Refinamiento de objetivos y requerimientos (2h)
\item Análisis de riesgo
\item Reconocimiento de casos de uso (3h)
\item División de CU en iteraciones según prioridad (1h)
\item Estimación de tiempos de CU (1h)
\item Análisis de escenarios y atributos de calidad del sistema (2h)
\item Diseño de arquitectura (8h)
\item Realización de tareas de CU1 (30h)
\item Realización de tareas de CU2 (13h)
\item Realización de tareas de CU3 (23h)
\item Realización de tareas de CU4 (40h)
\end{enumerate}
\end{itemize}

\subsection{Plan de Proyecto}
En lo que sigue, se presenta el diagrama de Gantt con la planificación estimada para el desarrollo de la primera iteración, tal como se detalló antes. Se asignaron los 4 recursos disponibles (desarrolladores) de manera tal que, cuando fuera posible, se paralelizaran actividades dividiendo el trabajo de tareas independientes. Debido a la alta carga de desarrollo requerida por las tareas de los casos de uso, lo cual no se lleva a cabo en la realización real de este trabajo, se asumió que cada desarrollador cuenta con una dedicación de 8 horas para el proyecto. Luego, la duración de las tareas está expresada en la proporción de días correspondientes a esas jornadas de trabajo. Vale observar que si bien la extensión gráfica de la duración de las tareas se superpone con fines de semana, éstos no han sido contemplados como días laborables para el cálculo de las semanas.
\newpage

%%%%%%%%%%%%%%%%%%%%%%%%%%%%%%%%%%%%%%%%%%%%%%%%%%%%%%%%%%%%%%%%%%%%%%%%%%%%%%%
%% Análisis de riesgo                                                        %%
%%%%%%%%%%%%%%%%%%%%%%%%%%%%%%%%%%%%%%%%%%%%%%%%%%%%%%%%%%%%%%%%%%%%%%%%%%%%%%%

\section{Análisis de riesgos}
\newpage

\end{document}
