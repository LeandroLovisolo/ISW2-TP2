\documentclass[a4paper, 10pt, twoside]{article}

\usepackage[top=1in, bottom=1in, left=1in, right=1in]{geometry}
\usepackage[utf8]{inputenc}
\usepackage[spanish, es-ucroman, es-noquoting]{babel}
\usepackage{setspace}
\usepackage{fancyhdr}
\usepackage{lastpage}
\usepackage{amsmath}
\usepackage{amsfonts}
\usepackage{amsthm}
\usepackage{verbatim}
\usepackage{fancyvrb}
\usepackage{graphicx}
\usepackage{float}
\usepackage{enumitem} % Provee macro \setlist
\usepackage{tabularx}
\usepackage{multirow}
\usepackage{hyperref}
\usepackage{lscape}
\usepackage{xspace}
\usepackage{qtree}
\usepackage[toc, page]{appendix}


%%%%%%%%%% Constantes - Inicio %%%%%%%%%%
\newcommand{\titulo}{Trabajo Práctico 2}
\newcommand{\materia}{Ingeniería de Software II}
\newcommand{\integrantes}{Izcovich · Lovisolo · Petaccio · Vita}
\newcommand{\cuatrimestre}{Primer Cuatrimestre de 2015}
%%%%%%%%%% Constantes - Fin %%%%%%%%%%


%%%%%%%%%% Configuración de Fancyhdr - Inicio %%%%%%%%%%
\pagestyle{fancy}
\thispagestyle{fancy}
\lhead{\titulo\ · \materia}
\rhead{\integrantes}
\renewcommand{\footrulewidth}{0.4pt}
\cfoot{\thepage /\pageref{LastPage}}

\fancypagestyle{caratula} {
   \fancyhf{}
   \cfoot{\thepage /\pageref{LastPage}}
   \renewcommand{\headrulewidth}{0pt}
   \renewcommand{\footrulewidth}{0pt}
}
%%%%%%%%%% Configuración de Fancyhdr - Fin %%%%%%%%%%


%%%%%%%%%% Miscelánea - Inicio %%%%%%%%%%
% Evita que el documento se estire verticalmente para ocupar el espacio vacío
% en cada página.
\raggedbottom

% Separación entre párrafos.
\setlength{\parskip}{0.5em}

% Separación entre elementos de listas.
\setlist{itemsep=0.5em}

% Asigna la traducción de la palabra 'Appendices'.
\renewcommand{\appendixtocname}{Apéndices}
\renewcommand{\appendixpagename}{Apéndices}

\newcommand{\diagrama}[1]{
  \begin{center}
    \includegraphics[width=16cm]{#1}
  \end{center}
}

\newcommand{\diagramadeancho}[2]{
  \begin{center}
    \includegraphics[width=#1]{#2}
  \end{center}
}

\newcommand{\riesgo}[7]{
  \underline{Riesgo {#1}:}
  \begin{itemize}   
    \item \textbf{Descripción:} {#2}
    \item \textbf{Probablidad:} {#3}
    \item \textbf{Impacto:} {#4}
    \item \textbf{Exposición:} {#5}
    \item \textbf{Mitigación:} {#6}
    \item \textbf{Plan de contingencia:} {#7}
  \end{itemize}
}

\newcommand{\escenario}[7] {
  \textit{{#1}}
  \begin{itemize}
    \item \textbf{Fuente:} {#2}
    \item \textbf{Estímulo:} {#3}
    \item \textbf{Entorno:} {#4}
    \item \textbf{Artefacto:} {#5}
    \item \textbf{Respuesta:} {#6}
    \item \textbf{Medición:} {#7}
  \end{itemize}
}

%%%%%%%%%% Miscelánea - Fin %%%%%%%%%%

\begin{document}


%%%%%%%%%%%%%%%%%%%%%%%%%%%%%%%%%%%%%%%%%%%%%%%%%%%%%%%%%%%%%%%%%%%%%%%%%%%%%%%
%% Carátula                                                                  %%
%%%%%%%%%%%%%%%%%%%%%%%%%%%%%%%%%%%%%%%%%%%%%%%%%%%%%%%%%%%%%%%%%%%%%%%%%%%%%%%


\thispagestyle{caratula}

\begin{center}

\includegraphics[height=2cm]{DC.png} 
\hfill
\includegraphics[height=2cm]{UBA.jpg} 

\vspace{2cm}

Departamento de Computación,\\
Facultad de Ciencias Exactas y Naturales,\\
Universidad de Buenos Aires

\vspace{4cm}

\begin{Huge}
\titulo
\end{Huge}

\vspace{0.5cm}

\begin{Large}
\materia
\end{Large}

\vspace{1cm}

\cuatrimestre

\vspace{4cm}

\begin{tabular}{|c|c|c|}
\hline
Apellido y Nombre & LU & E-mail\\
\hline
Izcovich, Sabrina      & 550/11 & sizcovich@gmail.com\\
Lovisolo, Leandro      & 645/11 & leandro@leandro.me\\
Petaccio, Lautaro José & 443/11 & lausuper@gmail.com\\
Vita, Sebastián        & 149/11 & sebastian\_vita@yahoo.com.ar\\
\hline
\end{tabular}

\end{center}

\newpage

\tableofcontents

\newpage


%%%%%%%%%%%%%%%%%%%%%%%%%%%%%%%%%%%%%%%%%%%%%%%%%%%%%%%%%%%%%%%%%%%%%%%%%%%%%%%
%% Introducción                                                              %%
%%%%%%%%%%%%%%%%%%%%%%%%%%%%%%%%%%%%%%%%%%%%%%%%%%%%%%%%%%%%%%%%%%%%%%%%%%%%%%%

\section{Introducción}

En el presente trabajo se aborda la extensión del diseño presentado en el TP1, debiendo ahora plantear un sistema de mensajes de muchas campañas a nivel nacional tanto para servicios públicos como privados. Dada esta nueva situación, aparecen varios stakeholders que, por su colaboración en el proyecto proveyendo servicios, o por su intención de consumo, imponen distintos requerimientos que sumados a los técnicos desembocan en la contemplación de diversos riesgos y atributos de calidad.
Aquí se presenta una planificación de las fases de elaboración, construcción y transición de la metodología UP, asumiendo que la fase inception ya se ha desarrollado. En esta primera etapa se presentan las ideas para la formulación del proyecto Big Tiza, se establecen los requerimientos y recursos disponibles, y se lleva a cabo el QAW del cual se desprendieron algunos atributos de calidad esperados en el producto final. Como parte de la primera iteración, se identifican los casos de uso más relevantes y en base a la recopilación de toda esta información se realiza un análisis de riesgos que se utiliza en parte para la organización de los CU en las distintas iteraciones incrementales. Además se diseña la arquitectura del sistema.

\newpage

%%%%%%%%%%%%%%%%%%%%%%%%%%%%%%%%%%%%%%%%%%%%%%%%%%%%%%%%%%%%%%%%%%%%%%%%%%%%%%%
%% Casos de uso                                                              %%
%%%%%%%%%%%%%%%%%%%%%%%%%%%%%%%%%%%%%%%%%%%%%%%%%%%%%%%%%%%%%%%%%%%%%%%%%%%%%%%

\section{Casos de uso}
Para comenzar a trabajar sobre la planificación del proyecto se identificó un conjunto de casos de uso que cubren la mayoría de las funcionalidades que se desprenden de los requerimientos obtenidos. En un principio se los agrupó, tal y como se muestra a continuación, por su funcionalidad. Posteriormente, y en base al análisis de riesgos y a la información destilada de la fase de incepción, se utilizó este refinamiento para la organización y planificación de las iteraciones de elaboración y construcción.

\begin{itemize}

\item \textbf{Comunicación}
\begin{itemize}
\item Enviando mensajes de campañas por medios seleccionados\footnote{Facebook, Twitter, Whatsapp o SMS.}.
\item Respondiendo mensaje a una pregunta.
\end{itemize}

\item \textbf{Monitoreo}
\begin{itemize}
\item Consultando estado de una campaña.
\item Visualizando el mapa de evolución de las campañas.
\end{itemize}

\item \textbf{Auditoría}
\begin{itemize}
\item Registrando cantidad de mensajes enviados por campaña.
\item Generando factura mensual para la empresa de infraestructura privada.
\end{itemize}

\item \textbf{Gestión}
\begin{itemize}
\item Ingresando los resultados de una campaña.
\item Creando, editando y eliminando grupos de destinatarios.
\item Creando una campaña desde el sector público/privado.
\item Desuscribiendo a un destinatario de una campaña.
%\item Reinscribiendo a un destinatario en una campaña de la que se desuscribió.
\item Agregando evaluación manual del resultado de las preguntas.
\item Aceptando campaña privada.
\item Creando cuestionario de campañas de evaluación de campañas.
\item Agregando nuevo usuario proveedor de contenidos.
\item Creando, editando y eliminando usuarios.
\end{itemize}

\item \textbf{Supervisión}
\begin{itemize}
\item Supervisando campañas privadas (supervisar encuestas)?
\item Pidiendo reportes de resultados de una campaña
\item Reportando información anonimizada de grupos de destinatarios.
\end{itemize}

\item \textbf{Seguridad}
\begin{itemize}
\item Autenticando usuario.
\item Guardando datos cifrados en servidores locales.
\end{itemize}
\end{itemize}

\newpage

%%%%%%%%%%%%%%%%%%%%%%%%%%%%%%%%%%%%%%%%%%%%%%%%%%%%%%%%%%%%%%%%%%%%%%%%%%%%%%%
%% Planificación                                                          %%
%%%%%%%%%%%%%%%%%%%%%%%%%%%%%%%%%%%%%%%%%%%%%%%%%%%%%%%%%%%%%%%%%%%%%%%%%%%%%%%

\section{Planificación}

\subsection{Iteraciones}
La división de los CU en las distintas iteraciones y la organización de las mismas estuvo pautada por las necesidades manifestadas por los stakeholders, el análisis de riesgo y un incremento de funcionalidad, partiendo del núcleo más importante y revistiendo la aplicación a lo largo del tiempo de lógica menos prioritaria.

\subsubsection{Fase de iniciación}
Consideramos fase de iniciación al trabajo realizado en el Primer Trabajo Práctico junto con la Arquitectura presente en este TP.

\subsubsection{Fase de elaboración}

\textbf{Primera iteración} [3 semanas]
\begin{enumerate}
\item Guardando datos cifrados en servidores locales.
\item Creando una campaña desde el sector público/privado.
\item Enviando mensajes de campañas por medios seleccionados.
\item Autenticando usuario.
\end{enumerate}

\textbf{Segunda iteración} [5 semanas]
\begin{enumerate}
\item Creando, editando y eliminando usuarios.
\item Consultando estado de una campaña.
\item Registrando cantidad de mensajes enviados por campaña.
\item Creando, editando y eliminando grupos de destinatarios.
\end{enumerate}

\textbf{Tercera iteración} [5 semanas]
\begin{enumerate}
\item Creando cuestionario de campañas de evaluación de campañas.
\item Respondiendo mensaje a una pregunta.
\item Desuscribiendo a un destinatario de una campaña.
\item Reinscribiendo a un destinatario en una campaña de la que se desuscribió.
\end{enumerate}

\textbf{Cuarta iteración} [4 semanas]
\begin{enumerate}
\item Ingresando los resultados de una campaña.
\item Pidiendo reportes de resultados de una campaña.
\item Agregando evaluación manual del resultado de las preguntas.
\end{enumerate}

\textbf{Quinta iteración} [4 semanas]
\begin{enumerate}
\item Aceptando campaña privada.
\item Reportando información anonimizada de grupos de destinatarios.
\item Generando factura mensual para el sector privado.
\item Enviando resultados de campañas a la NSA.
\end{enumerate}

\textbf{Sexta iteración} [4 semanas]
\begin{enumerate}
\item Supervisando campañas privadas (supervisar encuestas).
\item Visualizando el mapa de evolución de las campañas.
\item Agregando nuevo usuario proveedor de contenidos.
\end{enumerate}

\subsubsection{Fase de construcción}
En la fase de construcción se crea el producto. Para poder lograrlo, se van desarrollando los casos de uso de cada una de las iteraciones previamente definidas en la etapa de elaboración. Al finalizar el desarrollo de cada una de las iteraciones, se debe contar con un software funcional que responda a todas las especificaciones. Además se debe proveer de casos de prueba para poder constatar el correcto funcionamiento del programa.

\subsubsection{Fase de transición}
En esta fase, el sistema ya se encuentra funcionando en una versión $Beta$. Esto permite poder arreglar problemas que se encuentran en el funcionamiento y que no fueron detectados en la fase anterior. Por otro lado, se utiliza esta fase tanto para poder enseñarle a los usuarios como manejar el sistema, como para extender la funcionabilidad del mismo en el caso de que sea necesario.

\subsection{Alcance de casos de usos de la primera iteración}
\textbf{Guardando datos cifrados en servidores locales.}

Los datos de los usuarios deben almacenarse encriptados en los servidores locales de su región correspondiente para poder garantizar la confidencialidad de los mismos.

Llamamos datos a los registros en la base de datos que contienen información sobre los residentes de la región. Estos pueden ser datos básicos como por ejemplo (nombre, apellido, dni, número de teléfono) y datos adiciones como si es fumador, si poseen un auto, si les gusta correr, etc.

\textbf{Creando una campaña desde el sector público/privado.}

La creación de una campaña consiste en el ingreso de los datos necesarios para la creación de la misma, es decir, un conjunto de mensajes con el horario en el que deben ser enviados, un conjunto de grupos de destinatarios concernientes al objetivo de la campaña y el canal a través del que se enviarán (SMS, Facebook, etc).

\textbf{Enviando mensajes de campañas por medios seleccionados.}

Cada región tiene un sistema de envío de mensajes que periódicamente recorre las campañas asociadas a esa región y envía los mensajes de las mismas cumpliendo su fecha y hora de envío. Además, cada mensaje es enviado por los canales previamente seleccionados a través de la infraestructura propia de cada región.

\textbf{Autenticando usuario.}

Se solicita un usuario y contraseña cuando una persona intenta iniciar una sesión en el sistema para crear una campaña y consultar el estado de una campaña, entre otras. La autenticación también abarca al sector privado dónde tendran acceso sólo a acciones de su sector.

El sistema verifica que las credenciales ingresadas sean válidas, y en caso contrario deniega el acceso al mismo.

\subsection{Tareas CU Primera iteración}
A continuación se detallan las tareas diagramadas para los casos de uso incluidos en la primera iteración con su respectiva estimación de horas hombre.
\\

\begin{tabular}{lp{13cm}l}
  \hline
  CU1 & Guardando datos cifrados en servidores locales & 140h \\
  \hline
  T01 & Investigación de fuentes de información disponibles de cada usuario accesible desde el sistema & 16h \\
  T02 & Definición de campos relevantes para almacenar por usuario & 12h \\
  T03 & Diseño de la base de datos & 16h \\
  T04 & Definición de motor de búsqueda a utilizar & 8h \\
  T05 & Definición de método de encriptación & 8h \\
  T06 & Implementación de CU1 & 40h\\
  T07 & Testing de CU1 & 40h \\
  \hline
\end{tabular}

\vspace{1em}

\begin{tabular}{lp{13cm}l}
  \hline
  CU2 & Creando una campaña desde el sector público/privado & 42h \\
  \hline
  T01 & Investigar librerías y herramientas para la implementación de las campañas & 8h \\
  T02 & Diseño de la interfaz del usuario & 4h \\
  T03 & Implementación de CU2 & 15h \\
  T04 & Testing de CU2 & 15h \\
  \hline
\end{tabular}

\vspace{1em}

\begin{tabular}{lp{13cm}l}
  \hline
  CU3 &  Enviando mensajes de campañas por medios seleccionados & 160h \\
  \hline
  T01 & Investigación de los distintos medios de envío de mensajes & 40h \\
  T02 & Diseño de un servicio EmisorDeMensajes, que en intervalos de tiempo regulares envíe los mensajes programados & 40h \\
  T03 & Implementación de CU3 & 40h \\
  T04 & Testing de CU3 & 40h \\
  \hline
\end{tabular}

\vspace{1em}

\begin{tabular}{lp{13cm}l}
  \hline
  CU4 & Autenticando usuario & 34h \\
  \hline
  T01 & Diseño de la interfaz del usuario para autenticación & 4h \\
  T02 & Diseño de los requisitos mínimos de las contraseñas & 8h \\
  T02 & Implementación de CU4 & 11h \\
  T03 & Testing de CU4 & 11h \\
  \hline
\end{tabular}


\subsection{Detalle Primera iteración}

\begin{itemize}
  \item \textbf{Identificación:} E1
  \item \textbf{Tipo de iteración:} Elaboración
  \item \textbf{Cantidad total de horas:} 480
  \item \textbf{Tareas:}
\begin{enumerate}
  \item Refinamiento de objetivos y requerimientos (20h)
  \item Análisis de riesgo (12h)
  \item Reconocimiento de casos de uso (12h)
  \item División de CU en iteraciones según prioridad (10h)
  \item Estimación de tiempos de CU (10h)
  \item Análisis de escenarios y atributos de calidad del sistema (8h)
  \item Diseño de arquitectura (32h)
  \item Realización de tareas de CU1 (140h)
  \item Realización de tareas de CU2 (42h)
  \item Realización de tareas de CU3 (160h)
  \item Realización de tareas de CU4 (34h)
\end{enumerate}
\end{itemize}

\subsection{Plan de Proyecto}
En lo que sigue, se presenta el diagrama de Gantt con la planificación estimada para el desarrollo de la primera iteración, tal como se detalló antes. Se asignaron los 4 recursos disponibles (desarrolladores) de manera tal que, cuando fuera posible, se paralelizaran actividades dividiendo el trabajo de tareas independientes. Debido a la alta carga de desarrollo requerida por las tareas de los casos de uso, lo cual no se lleva a cabo en la realización real de este trabajo, se asumió que cada desarrollador cuenta con una dedicación de 8 horas para el proyecto. Luego, la duración de las tareas está expresada en la proporción de días correspondientes a esas jornadas de trabajo. Vale observar que si bien la extensión gráfica de la duración de las tareas se superpone con fines de semana, éstos no han sido contemplados como días laborables para el cálculo de las semanas.

\begin{landscape}
\begin{figure}[h!]
  \centering
  \includegraphics[width=20cm]{gantt.png}
  \caption{Diagrama de Gantt de la 1era iteración con la asignación de tiempo}
  \label{fig:gantt}
\end{figure}
\end{landscape}
\newpage

%%%%%%%%%%%%%%%%%%%%%%%%%%%%%%%%%%%%%%%%%%%%%%%%%%%%%%%%%%%%%%%%%%%%%%%%%%%%%%%
%% Análisis de riesgo                                                        %%
%%%%%%%%%%%%%%%%%%%%%%%%%%%%%%%%%%%%%%%%%%%%%%%%%%%%%%%%%%%%%%%%%%%%%%%%%%%%%%%

\section{Análisis de riesgos}
\label{riesgos:r1}
\riesgo{1}
    {Mantener la disponibilidad del servicio en la totalidad del país es importante. Sin embargo, se conoce que la disponibilidad del servicio puede tener dificultades en algunas zonas, como por ejemplo la patagonia.}
    {Alta}
    {Alto}
    {Alta}
    {Mantener un sistema que permita verificar la conexión de los distintos nodos que se encuentran en distintas regiones del país}
    {Al reportarse un fallo en la conexión, tercerizar la conexión a privados hasta que se arregle el problema}

\riesgo{2}
    {ArSat nos provee actualmente conexión y procesamiento en la región de Córdoba. Si se encuentra un problema en los nodos de ANSES, existirán problemas para procesar la información almacenada.}
    {Media}
    {Medio}
    {Medio}
    {Realizar backups de los nodos regionales con cierta frecuencia y almacenarlos de forma cifrada en una empresa tercerizada. Monitorear desde cada nodo el estado del resto de los nodos de las otras regiones mediante heartbeat.}
    {Notificar vía email al responsable técnico de un nodo al momento de detectarse su caída. Utilizar el backup cifrado en la empresa para lanzar una nueva instancia del nodo regional caído.}

\riesgo{3}
    {El modelo de abono que contabiliza los mensajes enviados está por migrar pronto a mejores servidores, por lo que se esperan algunas fallas de comunicación durante la transición.}
    {Alta}
    {Alto}
    {Alta}
    {Mantener un sistema de respaldo durante la migración del servicio para así poder asegurar la comunicación.}
    {Utilizar el sistema de respaldo para mantener la mayor correctitud posible los balances en el modelo de abonos a privados.}

\newpage

\section{Atributos de calidad}

Los atributos de calidad principales que debe respetar el sistema a realizar, en orden de prioridad, son los que siguen:

\begin{enumerate}
\item Disponibilidad
\item Performance
\item Seguridad
\item Certeza de Datos
\item Modificabilidad
\item Flexibilidad
\item Usabilidad
\end{enumerate}


Para describirlos, decidimos detallar escenarios correspondientes a cada uno de ellos.
Cada escenario está planteado en base al QAW y a los requerimientos o peticiones realizadas por los stakeholders en el enunciado.

\subsection{Disponibilidad}
\label{disponibilidad:atr1}
\escenario{El sistema debe ser capaz de soportar fallas de los distintos enlaces de comunicación.}
    {Interna.}
    {Se produce una falla en la conexión.}
    {Operación degradada.}
    {Conexiones del sistema.}
    {Se cambia rápidamente a otro tipo de conexión.}
    {Se tarda a lo sumo 10 minutos en cambiar a otro enlace de comunicación.}

\escenario{Se quiere poder agregar nuevas campañas en todo momento.}
    {Interna.}
    {Un usuario desea agregar una campaña en el sistema.}
    {Operación normal.}
    {Sistema.}
    {El sistema permite crear una campaña en todo momoento.}
    {Cada intento de creación de campaña es exitoso bajo operación normal.}

\subsection{Modificabilidad}
\label{modificabilidad:atr1}
\escenario{Y pensar en ampliar los canales de comunicación además del SMS sumar Twitter, Whatsapp, Facebook Messenger, etc.}
    {Interna.}
    {Se desea extender el sistema.}
    {Operación normal.}
    {Sistema.}
    {El sistema es extensible a canales de comunicación en el menor tiempo posible.}
    {La extensión planteada no toma más de 8hs hombre modificando módulos relacionados.}


\subsection{Certeza de datos}
\escenario{Como el otorgamiento de información precisa en tiempo real es imposible (aún con el equipo necesario), se debe proveer datos estimados.}
    {Interna.}
    {Un usuario desea ver la evolución de una campaña.}
    {Operación normal.}
    {Sistema.}
    {Se presentan datos estimados de la campaña, lo más precisos posibles.}
    {Los datos presentados tienen una antiguedad de a lo sumo 10 minutos.}

\subsection{Seguridad}
\label{seguridad:atr1}
\escenario{Quiere asegurar confidencialidad de todos los datos involucrados.}
    {Externa.}
    {Un individuo malintencionado intenta interferir los mensajes que se envían entre enlaces.}
    {Online.}
    {Sistema.}
    {Se utilizan cifrados para los mensajes los cuales tardarían un tiempo alto en romperse sin las claves correspondientes.}
    {Se tarda más de 10 mil años en realizar la decodificación sin claves.}

\escenario{Sólo personal autorizado y sus delegados en cada provincia tengan acceso a definir las campañas, y evolución.}
    {Externa.}
    {Un usuario sin privilegios intenta acceder a datos sobre campañas o definir una.}
    {Online.}
    {Sistema.}
    {Se utiliza un sistema de autenticación basado en usuarios y contraseñas. Se exigen contraseñas fuertes mediante restricciones de longitud mínima, combinación de caracteres alfanuméricos y especiales, etc.}
    {En las auditorías de seguridad sólo se logran adivinar las contraseñas de menos del 1\% de las cuentas de usuario.}

\escenario{Todos los datos que se manejarán son sensibles, y debe protegerse su acceso.}
    {Externa.}
    {Un individuo malintencionado intenta acceder a los datos guardados en las regiones o en el sistema de abono de datos.}
    {Online.}
    {Sistema.}
    {Se utilizan cifrados para los datos almacenados que tardarían un tiempo alto en decodificarse sin las claves correspondientes.}
    {Se tarda más de 10 mil años en realizar la decodificación sin claves.}

\subsection{Performance}

\escenario{Se debe poder monitorear el estado de las campañas de manera ágil. No se deben admitir demoras de ningún tipo.}
    {Externa.}
    {Un usuario desea ver el estado de una campaña.}
    {Operación normal.}
    {Sistema.}
    {Se visualiza el estado de la campaña.}
    {Se obtiene un reporte visual del estado de la campaña en menos de 10 segundos.}

\escenario{Se quiere que distintas empresas puedan tener acceso rápidamente a los datos de evaluación para mejorar los perfiles de marketing.}
    {Interna.}
    {Una empresa desea acceder a los datos de evaluación.}
    {Operación normal.}
    {Sistema.}
    {Se reportan los datos de evaluación de las campañas solicitadas.}
    {Se obtiene un reporte del estado de evaluación de las campañas solicitadas en menos de 10 segundos.}

\subsection{Usabilidad}

\escenario{Se quiere que el sistema sea fácil de usar para todos aquellos interesados en promover campañas de todo tipo.}
    {Interna.}
    {Un usuario desea poder utilizar fácilmente el sistema para crear campañas.}
    {Operación normal.}
    {Sistema.}
    {Los usuarios pueden utilizar fácilmente el sistema.}
    {Los usuarios tardan menos de 30 minutos en aprender a crear campañas en el sistema.}

\newpage

\section{Arquitectura BigTiza}

\subsection{Diagrama de componentes y conectores}
En lo que sigue, presentamos la arquitectura del trabajo con vista de Componentes y Conectores. Para la realización de la misma, utilizamos conectores creados específicamente para nuestras necesidades. Éstos son:

\begin{figure}[h!]
  \diagramadeancho{12cm}{./diagramas/referenciadeconectores.pdf}
  \caption{Referencia de los conectores creados.}
\end{figure}

\begin{itemize}
  \item \textbf{Conector encriptado asíncrono:} conector asíncrono de la cátedra con cifrado.
  \item \textbf{Conector selector asíncrono:} funciona como un proxy, redirige los datos por el sistema de ArSat o por el sistema de abono de datos hasta su destino. Prioriza el sistema de ArSat, si este no está disponible utiliza el sistema de abono de datos. Guarda la cantidad de datos que se envían por el sistema de abono de datos para llevar auditoría de lo utilizado. Los datos siempre viajan cifrados. Se encuentra definido a posteriori.
  \item \textbf{Conector asíncrono:} conector clásico de la cátedra.
  \item \textbf{Conector encriptado client/server:} conector client/server de la cátedra con cifrado.
  \item \textbf{Conector selector client/server:} conector selector definido anteriormente con capacidad client/server.
  \item \textbf{Conector síncrono:} conector clásico de la cátedra.
  \item \textbf{Conector repositorio encriptado:} conector clásico de la cátedra con cifrado.
  \item \textbf{Conector client/server:} conector clásico de la cátedra.

\end{itemize}

\newpage

\subsection{Diagrama de componentes y conectores}


\subsubsection{Nodo regional}

\diagramadeancho{14cm}{./diagramas/nodoregional.pdf}
\newpage


\subsubsection{Actuador web}

\diagrama{./diagramas/actuadorweb.pdf}
\newpage


\subsubsection{Emisor de mensajes}

\diagrama{./diagramas/emisordemensajes.pdf}
\newpage


\subsubsection{Receptor de mensajes}

\diagramadeancho{12cm}{./diagramas/receptordemensajes.pdf}
\newpage


\subsubsection{Conector selector (proxy)}

\diagrama{./diagramas/conectorselector.pdf}
\newpage


\subsection{Justificación de la arquitectura}

Para la realización de la arquitectura correspondiente al Sistema BigTiza se tuvieron en cuenta, principalmente, los atributos de calidad descritos anteriormente. 

Elegimos un estilo de arquitectura distribuida, es decir, no existe un sistema centralizado, existen nodos regionales los cuales resuelven las necesidades a nivel sistema de cada región y se comunican con los demás nodos para obtener información de ellos. Esta elección se hace en pos de facilitar la escalabilidad. Agregar un nuevo nodo no tiene un gran impacto en los demás. 

En lo que sigue, explicamos la arquitectura realizada.

\subsubsection{Nodo regional}

\begin{itemize}
  \item El \textit{servidor web} sirve la interfaz de usuario y atiende pedidos de operaciones, que son delegadas al \textit{actuador web}.

  \item El \textit{actuador web} se encarga de responder todas las operaciones solicitadas por la interfaz. Para esto delega muchas responsabilidades a otros componentes y repositorios del nodo regional. Además es quien se encarga de interactuar con otros nodos regionales, comunicándose con sus respectivos actuadores web.

  \item El \textit{autenticador} recibe pedidos del \textit{actuador web} para identificar usuarios a partir del par usuario/contraseña ingresado. Para esto se comunica con el \textit{repositorio de usuarios}, que almacena dicha información.

  \item La evaluación de la efectividad de las campañas es coordinada por el componente \textit{actualizador de datos de evaluación de campaña}. Hay tres formas por las que se evalúa una campaña:

  \begin{itemize}
    \item Por medio de la interfaz web. Para esto el \textit{actuador web} se comunica con el \textit{actualizador de datos de evaluación de campaña} para actualizar una campaña con los datos ingresados manualmente por un usuario desde la interfaz web.

    \item Por medio de las respuestas recibidas a mensajes de evaluación, que son recibidas desde el componente \textit{receptor de mensajes}.

    \item Por medio de un \textit{sistema de datos de evaluación} externo.
  \end{itemize}

  Los datos recogidos por los medios anteriores se almacenan en el \textit{repositorio de datos de evaluación de campañas}.

  \item El actuador web responde consultas de estado y evaluación de campañas leyendo el \textit{caché de estado de campañas} y \textit{caché de evaluación de campañas}. En estos cachés se guardan resultados de consultas frecuentes que de otra forma se realizarían contra los repositorios correspondientes, y la antigüedad de estos datos nunca supera un máximo de minutos preestablecido.

  \item El \textit{actualizador de backups} consulta todos los repositorios del nodo periódicamente y almacena backups de los mismos en servidores externos provistos por el \textit{sistema de abono de datos}.

  \item El emisor de mensajes consulta periódicamente el \textit{repositorio de campañas} y el \textit{system timer} para decidir qué mensajes deben ser enviados en ese momento, consulta los destinatarios de los mismos al \textit{repositorio de grupos de personas} y realiza el envío de los mismos por el servicio que corresponda (\textit{servicio de email}, \textit{servicio de SMS} o \textit{servicio de Whatsapp}.)

  \item El \textit{receptor de mensajes} consulta estos servicios para obtener potenciales respuestas a los mensajes de evaluación asociados a las campañas, y en caso de recibir alguno, notifica al \textit{actualizador de datos de evaluación de campañas}.

  \item Finalmente, el \textit{repositorio de uso de datos} es actualizado por todos los conectores de tipo selector cada vez que se envían datos por medio del proveedor de abono de datos para contabilizar los datos transmitidos y poder verificar que se esté facturando el tráfico que realmente corresponda.
\end{itemize}


\subsubsection{Receptor de mensajes}

A grandes rasgos, el componente \textit{Receptor de mensajes} tiene la responsabilidad de recibir los mensajes entrantes correspondientes a las respuestas de las preguntas realizadas para la evaluación de las campañas.

El componente cuenta con la siguiente estructura:

\begin{itemize}
  \item Un bus publish/subscribe con tres componentes de recepción de mensajes vía SMS, Whatsapp y email que se comunican con los servicios externos correspondientes y publican los mensajes en el bus a medida que llegan. La motivación del bus publish/subscribe también es permitir la facilidad de cambio a la hora de agregar nuevos canales de envío y recepción de mensajes, cumpliendo con el atributo de modificabilidad mencionado en \ref{modificabilidad:atr1}.

  \item Un componente \textit{Director de mensajes} el cual se subscribe a al bus descripto para recibir los mensajes o respuestas recibidas de distintos medios. Este componente procesa las respuestas para que el \textit{Actualizador de datos de evaluación de campañas} pueda consumirlas, pudiendo posteriormente realizar las actualizaciónes correspondientes.

\end{itemize}

\subsubsection{Emisor de mensajes}

El componente \textit{Emisor de mensajes} cuenta con los siguientes componentes:

\begin{itemize}
  \item Un \textit{Notificador de mensajes a enviar}, que consulta periódicamente al \textit{Repositorio de campañas} auxiliado del \textit{System timer} para decidir los mensajes que requieren ser enviados, y decide los destinatarios de los mismos consultando al \textit{Repositorio de grupos de personas}. Una vez conocidos los mensajes y los destinatarios, se publican los mismos en el bus publish/subscribe anterior.

  \item Un bus publish/subscribe, al que subscriben tres componentes: \textit{Emisor de mensajes via email}, \textit{Emisor de mensajes via Whatsapp} y \textit{Emisor de mensajes via SMS}. Éstos toman los mensajes que les corresponden del bus para efectuar el envío de los mensajes mediante los distintos servicios externos. Se utiliza un bus publish/subscribe para permitir fácilmente y con impacto mímimo la incorporación de nuevos emisores para nuevos canales de envío en el futuro, satisfaciendo el atributo de modificabilidad planteado anteriormente en \ref{modificabilidad:atr1}.
\end{itemize}

\subsubsection{Actuador web}

\begin{itemize}
\item El Receptor de mensajes web se encarga de aceptar los mensajes provenientes de la interfaz web. El mismo cataloga a qué componente le corresponde atender el mensaje recibido.
\item El Administrador de grupos de personas se encarga de seleccionar a las personas afectadas por la petición recibida. El mismo busca los usuarios necesarios en el Repositorio de grupos de personas, que se encuentran previamente fraccionados por distintos criterios (por ejemplo: fumador, no fumador).
\item El Receptor de mensajes heartbeat se encarga de recibir las señales heartbeat de los distintos nodos pertenecientes al sistema. En el caso en el que alguno de ellos no fuera recibido, se envía la notificación al Servicio mail. Por lo tanto, el Receptor de mensajes es también el encargado de verificar que todos los nodos hayan enviado su señal heartbeat.
\item El Emisor de señales heartbeat envía cada cierta frecuencia un ``latido'' que recibirán los receptores de señales heartbeat del resto de los nodos.
\item El Receptor de datos de evaluación manual recibe resultados medidos manualmente por los usuarios para guardarlos en el Actualizador de datos de evaluación de campañas. Por ejemplo, si un usuario desea almacenar en el sistema la cantidad de asistencias a una carrera o cantidad de aprobados en un examen, lo debe ingresar manualmente en el sistema.
\item El Administrador de campañas recibe los mensajes del Receptor de mensajes web que corresponden a la gerencia de campañas. Entre las peticiones posibles, se encuentran la creación, modificación y supresión de campañas como también la provisión de resultados de las mismas. Dicha información se encuentra almacenada en los Repositorios correspondientes a las campañas. Por otro lado, el Administrador se comunica con otros nodos regionales para proveerles información sobre las campañas en caso de que lo requieran.
\item El Reportador de resultado de campaña responde los petitorios correspondientes a los resultados que se generan en base al Repositorio de datos de campañas (que conserva la totalidad de mensajes recibidos y los datos que puedan servir para la evaluación de los resultados). Dichos resultados son valores que se actualizan en el Caché de evaluación de campañas.
\end{itemize}

\subsubsection{Conector selector}

El conector selector funciona como intermediario entre un componente \textit{Emisor de mensajes} y otro \textit{Componente receptor}.

Este conector está compuesto por un \textit{Proxy de conexión} el cual se encarga de hacer llegar la información enviada entre el emisor y el receptor utilizando los medios disponibles. Para esto, verifica si el sistema de ArSat esta caído, si no lo estuviera, utiliza este medio para enviar la información, en caso contrario, utiliza el \textit{Sistema de abono privado}. De esta manera, garantizamos el atributo de disponibilidad planteado anteriormente en \ref{disponibilidad:atr1}.

Por otro lado, el componente se encarga de actualizar en el \textit{repositorio de uso de datos}, la cantidad de información que fué enviada a traves del \textit{sistema de abono privado}. Esto es utilizado como solución a la desconfianza del \textit{Sistema de abono privado} a la hora de facturar el uso de su sistema y al riesgo observado en \ref{riesgos:r1}.

Toda la información que se envía mediante este conector es cifrada para garantizar que la conexíón sea segura, cumpliendo así con el atributo de seguridad planteado en \ref{seguridad:atr1}.

\newpage


%%%%%%%%%%%%%%%%%%%%%%%%%%%%%%%%%%%%%%%%%%%%%%%%%%%%%%%%%%%%%%%%%%%%%%%%%%%%%%%%
% Arquitectura del TP 1                                                        %
%%%%%%%%%%%%%%%%%%%%%%%%%%%%%%%%%%%%%%%%%%%%%%%%%%%%%%%%%%%%%%%%%%%%%%%%%%%%%%%%


\section{Arquitectura del TP 1}

Para la arquitectura del TP 1, nos basamos en el diagrama de clases realizado considerando las decisiones tomadas a lo largo de la elaboración del mismo.

\begin{itemize}
\item El \textit{Servidor web} representa a la interfaz que se le muestra al usuario cuando desea ingresar al sistema.
\item El \textit{Receptor de peticiones web} se encarga del manejo de los pedidos realizados por los usuarios. Éstas pueden incluir información sobre las mediciones de eficacia, sobre el estado de las campañas, agregado o supresión de mensajes, etc.
\item El \textit{Autenticador} verifica el usuario y la contraseña ingresados por el usuario para otorgarle los permisos que le corresponden.
\item El \textit{Repositorio de credenciales} envía la información al \textit{Autenticador}, quien almacena la información de los directivos y maestros.
\item El \textit{Repositorio de mensajes} almacena los mensajes de las campañas. Funciona como repositorio (cuando un mensaje debe ser enviado al emisor de mensajes) y como blackboard (cuando un usuario agrega un nuevo mensaje).
\item El \textit{Repositorio de eventos} tiene guardados los eventos de las campañas y tiene el mismo rol que el repositorio de mensajes.
\item El \textit{Emisor de mensajes} es el encargado de transmitir los mensajes en tiempo y hora junto con los destinatarios de los mismos al \textit{Servicio de SMS} para que éste los envíe.
\item Para la realización de esto último, el \textit{Repositorio de alumnos} (que incluye a los padres de los mismos) le provee los datos necesarios al \textit{Emisor de mensajes} para que se les envíe el mensaje a los alumnos correspondientes.
\end{itemize}

El diagrama resultante es el siguiente:

\diagrama{./diagramas/arquitecturaTP1.png}

\newpage


%%%%%%%%%%%%%%%%%%%%%%%%%%%%%%%%%%%%%%%%%%%%%%%%%%%%%%%%%%%%%%%%%%%%%%%%%%%%%%%%
% Comparación entre arquitecturas de TPs 1 y 2                                 %
%%%%%%%%%%%%%%%%%%%%%%%%%%%%%%%%%%%%%%%%%%%%%%%%%%%%%%%%%%%%%%%%%%%%%%%%%%%%%%%%


\section{Comparación entre arquitecturas de TPs 1 y 2}

Como primer diferencia entre las arquitecturas del TP 1 y TP 2, podemos observar un importante incremento en la complejidad de las mismas al pasar de un sistema de un único servidor en una escuela a uno de gran escala capaz de soportar diferentes tipos de mensajes y distribuido en diferentes lugares del país.

A grandes rasgos, la apariencia de ambas arquitecturas es similar; hay repositorios de usuarios y mensajes, un servidor web para acceder al sistema y servicios de mensajes que se encargan del envío. Sin embargo, el sistema del TP2 se adapta a grandes escalas y amplias tecnologías que obligan a considerar una mayor cantidad de servicios, control del resto de los sistemas, múltiples conexiones y backups de los datos.

Dado que el sistema del TP1 funciona en una única computadora, no nos pareció necesaria la utilización de paralelismo. En cambio, los nodos regionales deben atender múltiples pedidos simultáneamente y procesar datos e información a una velocidad razonable, por lo que se necesita concurrencia de los módulos. 

Por otro lado, en el TP2 debimos asegurar seguridad de los datos, por lo que los conectores y los repositorios que se encuentran dentro de cada nodo están cifrados.

Como el sistema del TP2 está pensado para funcionar a nivel nacional, tuvimos que tener en cuenta detalles como la disponibilidad de conexión. Para esto, armamos un plan de respaldo en el caso de que la comunicación entre los distintos nodos se llegase a perder. Dicho plan contempla la posibilidad de usar servicios provistos por terceros. En cambio, el sistema del TP1 no contempla esto dado que es un sistema muy pequeño que funciona en una computadora.

\newpage


%%%%%%%%%%%%%%%%%%%%%%%%%%%%%%%%%%%%%%%%%%%%%%%%%%%%%%%%%%%%%%%%%%%%%%%%%%%%%%%%
% Conclusiones                                                                 %
%%%%%%%%%%%%%%%%%%%%%%%%%%%%%%%%%%%%%%%%%%%%%%%%%%%%%%%%%%%%%%%%%%%%%%%%%%%%%%%%


\section{Conclusiones}

A modo de conclusión, podemos determinar que, considerando el desarrollo realizado a lo largo de ambos tps, la comparación que podemos realizar entre el Scrum y el UP no llega a ser lo suficientemente estricta dado que no realizamos la implementación completa de los sistemas. Esto se debe a que con Scrum realizamos una única iteración y con UP no desarrollamos nada específico.

Sin embargo, podemos establecer una comparación entre las etapas iniciales de realizar una planificación y un diseño antes de la implementación del sistema, y comenzarla sin planificación alguna, diseñando incrementalmente.

Principalmente, descubrimos que el orden de importancia de los atributos de calidad es el principal dirigente en lo que concierne las decisiones arquitectónicas. En nuestro caso, tuvimos que priorizar la disponibilidad, por lo que el sistema debió verse afectado por distintos métodos capaces de asegurar la supervivencia del sistema más allá de las caídas.

Por otro lado, nos pareció importante la organización de tiempos y responsabilidades a través de un diagrama de Gantt para lograr precisión en la estimación de creación de un sistema. Notamos que un error en la estimación de esfuerzo en Scrum no presenta grandes problemas en comparación con las que tendría el mismo error (de estimación de horas/hombre) en UP. En el peor caso, en Scrum una historia quedará afuera de una iteración y tendrá que ser subdividida más adelante, en cambio, en UP se puede producir un efecto dominó que afecta todas las iteraciones posteriores y retrasar todo el proyecto.

El pasaje de ``programming in the small'' a ``programming in the large'' afecta proporcionalmente a la cantidad de componentes a considerar y conexiones a establecer, teniendo en cuenta los problemas que este agregado acompañan. Por otro lado, el análisis previo requiere encontrarle soluciones a nuevos problemas que un sistema simplificado no presentaría, como ser las conexiones y actualizaciones de datos entre los distintos nodos.
\end{document}
